% Chapter 1

\chapter{DEPAROM Profil und die Entstehung von Profil.NG} % Chapter title

\label{ch:profil.ng} % For referencing the chapter elsewhere, use \autoref{ch:introduction}

%----------------------------------------------------------------------------------------

Die vorliegende Arbeit entstand bei der Moving Targets Consulting GmbH (MTC) und
befasst sich mit der vollständigen Reimplementierung des Produktionssystems des
Produktes DEPAROM Profil. Bei DEPAROM Profil handelt es sich um ein
Patentinformationssystem, welches die wöchentlich von den verschiedenen
Patentämtern der Welt publizierten Patentschriften erfasst, normalisiert,
durchsuchbar macht und für die Kunden der MTC nach bestimmten Kriterien
überwacht. Patentschriften, welche in die Überwachung passen, werden in einer
normalisierten Form zusammengestellt und auf einem Datenträger dem Kunden
zugestellt. Die Kunden bezahlen MTC für diese Dienstleistung.

DEPAROM Profil wurde ursprünglich von der Bundesdruckerei GmbH Mitte der 1990er
Jahre entwickelt und sollte 2005 auslaufen. Es wurde an MTC inklusive der
Bestandskunden verkauft, fortgeführt und weiterentwickelt. Die Codebase aus C,
C++, Java, Shell- und Pythonscripten, D-Modulen und Perlscripten ist
mittlerweile zwanzig Jahre alt und in äußerst desolatem Zustand. Die Wartung
nimmt den größten Teil der verfügbaren Entwicklungsressourcen ein und viele
Bereiche des Systems bilden für die beteiligten Entwickler eine Blackbox.

Damit MTC auch in Zukunft konkurrenzfähig im Bereich der
Patentinformationssysteme bleibt, wurde deshalb durch die Geschäftsführung die
vollständige Neuimplementierung des Systems beschlossen.

Gegenstand dieser Arbeit soll ein Teilsystem von DEPAROM Profil sein, die
sogenannte Profilanwendung. Diese ist die Steuersoftware mit der die Produktion,
der Datenimport und die Kundenverwaltung gesteuert wird. Die neue Version dieser
Anwendung wird im Folgenden als Profil.NG bezeichnet und ist Teil des Systems
DEPAROM.NG.

Bei der Neuimplementierung sollte ein Ökosystem aus unabhängigen Diensten
entstehen, welche durch die zentrale Anwendung Profil.NG angesprochen werden, um
die Patentinformationen für die Kunden zu erstellen. Die einzelnen Dienste
kommunizieren über REST-Schnittstellen.

Bei der Realisierung kommen die Technologien Java Spring Boot \cite{spring-boot}
und Elasticsearch \cite{elasticsearch} für die Erstellung der Dienste zum
Einsatz und MongoDB \cite{mongodb} sowie die Javascript-Plattform Meteor
\cite{meteor} für die Entwicklung von Profil.NG.

Als besondere Herausforderung gilt die Analyse- und Entwurfsphase, da alle
bestehenden Geschäftsprozesse auf ihre Validität geprüft werden müssen, damit
eine fundierte Entscheidung darüber getroffen werden kann, ob sie übernommen,
modifiziert oder verworfen werden müssen.

Insbesondere interessant stellt sich die Verwendung von Meteor als Plattform
dar. Die vollständige Reaktivität der Plattform ermöglicht das Erforschen neuer
Entwurfsmuster, sowie die Herausforderung, die bewährten Muster aus dem
Standardwerk "Design Patterns. Elements of Reusable Object Oriented Software"
\cite{design-patterns} auf das ungewohnte Programmierparadigma anzuwenden.
