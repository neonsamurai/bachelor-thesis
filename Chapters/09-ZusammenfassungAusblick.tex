% Chapter 1

\chapter{Zusammenfassung und Ausblick} % Chapter title

\label{ch:zusammenfassungUndAusblick} % For referencing the chapter elsewhere, use \autoref{ch:introduction}

%----------------------------------------------------------------------------------------


Im Verlauf dieser Bachelorarbeit konnten die Anforderungen an das System
\textit{Profil.NG} aus dem Vorgängersystem \textit{Profilanwendung} extrahiert,
bereinigt und formuliert werden. Aus den Anforderungen  ließ sich ein für die
Implementierung verwendbar Software-Entwurf ableiten, welcher für die
Entwicklung des vorliegenden Prototyps benutzt wurde.

Die Analyse- und Entwurfsphase fiel dabei überraschend groß aus. Es war
schwierig, die Anforderungen aus dem Vorgängersystem in einer ausreichenden
Abstraktion herauszulösen und für Profil.NG zu formulieren. Erschwerend kamen
Faktoren hinzu, die nicht im Bereich des Software-Engineering liegen. So waren
die für die Ermittlung der Geschäftsprozesse nötigen Personen nicht immer
zeitnah zu sprechen und es war zum Teil schwierig ihnen die Wichtigkeit ihrer
Rolle begreiflich zu machen. Hieraus resultierten zunächst unscharfe
Anforderungen, die sich im weiteren Verlauf des Projekts auch änderten. Hieraus
resultierten Verzögerungen, die dazu führten, dass verhältnismäßig wenig
Implementierungszeit zu Verfügung stand.

Das gewählte Entwicklungsverfahren der testgetriebenen Entwicklung erwies sich
zugleich als herausragend nützliches Werkzeug und als stärkste Bremse während
der Implementierungsphase. Die besondere Denkweise bei der testgetriebenen
Entwicklung und die geringe Erfahrung sowohl mit der Testbibliothek Jasmine als
auch in der testgetriebenen Entwicklung im Allgemeinen und mit Meteor im
speziellen erwiesen sich immer wieder als Implementierungsbremsen, da hier viel
Forschungsarbeit geleistet werden musste. Da man bei der Implementierung aber
gezwungen wird, sich zunächst über den gewünschten Effekt Gedanken zu machen und
nicht gleich in Implementierungsdetails verrennt, erlangt man ein besseres
Verständnis über die logische Problematik der Implementierung. Zudem kann man
sich in der testgetriebenen Entwicklung relativ sicher sein, keine
Implementierungsdetails vergessen zu haben, sobald alle Tests erfolgreich
verlaufen.

In Ermangelung von ausreichend Implementierungszeit ist vor allem noch die
Kommunikation mit den anderen Webdiensten im DEPAROM.NG-Gesamtsystem eine
ungelöste Aufgabe. Meteor liefert mit \textit{http} eine Bibliothek, die für
exakt diese Aufgabe entwickelt wurde. Eine Realisierung anhand von \textit{http}
scheint daher keine besonders schwierige Aufgabe.

Ein Problem ganz anderer Art ist die Schemavalidierung. Ein weitgehend
ungelöstes Problem im Meteor-Ökosystem ist die Validierung von Eingabedaten
gegen ein Dokumentschema. Durch die Definition solcher Schemata wäre es möglich,
solche Validierungen in einer standadisierten Art und Weise durchzuführen. Auch
wäre es möglich, Formulare für CRUD-Operationen programmatisch aus solchen
Schemata zu generieren. Während der Bearbeitungszeit für diese Arbeit stieß ich
leider erst recht spät auf eine Bibliothek, welche die Definition einfacher
Schemata ermöglicht. Die Verwendung selbiger vermied ich, da mein Zeitplan
ohnehin schon sehr knapp bemessen war.

Für die Zukunft soll eine Schemadefinition in jedem Fall hinzugefügt werden.

Im weiteren Verlauf des Projekts Profil.NG außerhalb dieser Bachelorarbeit soll
die testgetriebene Entwicklung in jedem Falle weitergeführt werden. Die
Anwendung wird für die Generierung von mehreren hunderttausend Euro Umsatz im
Jahre die zentrale Rolle spielen, weshalb eine reibungslose Weiterentwicklung zu
gewährleisten ist. Ich ziehe auch in Erwägung die vorhandenen Testkategorien
Unit- und Integrationstests noch um Akzeptanztests zu erweitern. Als mögliche
Technologien zur Realisierung sehe ich hier die Frameworks Cucumber oder
Nightwatch. Eine genauere Evaluierung der beiden Technologien sollte die Eignung
für Meteor allerdings zuvor feststellen.
