% Chapter 1

\chapter{Systementwurf} % Chapter title

\label{ch:Systementwurf} % For referencing the chapter elsewhere, use \autoref{ch:introduction}

%----------------------------------------------------------------------------------------

Zur besseren Einordnung von Profil.NG in das Gesamtsystem DEPAROM.NG sei hier
nochmals auf die  Systemübersicht (\autoref{fig:DNGOverview}, Seite
\pageref{fig:DNGOverview}) verwiesen.

Im folgenden Teil sollen die für Profil.NG verwendeten Technologien, die
Motivation hinter deren Verwendung und spezifische Entwurfsprobleme in Bezug auf
den Technologiestack näher beleuchtet werden.

\section{Systemarchitektur}

Das System Profil.NG besteht aus

\begin{itemize}
  \item{einer Clientkomponente}
  \item{einer Serverkomponente}
  \item{einer Datenbankkomponente}
\end{itemize}

Die Clientkomponente wird die Aufgaben der Benutzerschnittstelle übernehmen. Sie
soll als Fat-Client konzipiert in einem modernen Webbrowser laufen. Aufgrund der
Betriebsumgebung in einem Webbrowser ist die gewählte Programmiersprache
natürlicherweise JavaScript.

Die Kommunikation zwischen dem Client und dem Server wird über Websockets
ablaufen. Das Protokoll für die Datenübertragung wird das Distributed Data
Protcol \cite{ddp} sein.

Die Serverkomponente beherbergt die Geschäftslogik und ist für die Kommunikation
mit der Datenbank zuständig. Auch das Serverprogramm wird in JavaScript
geschrieben.

Das Datenbanksystem wird MongoDB sein, eine nichtrelationale, dokumentbasierte
Datenbank. Sie akzeptiert JSON-Dokumente, das Datenaustauschformat von
JavaScript, als Eingabedaten und eignet sich daher gut für ein System, welches
mehrheitlich in JavaScript geschrieben ist.

\begin{figure}[h]
  \includegraphics[width=0.9\textwidth]{gfx/meteor_architektur.pdf}
  \caption{Vereinfachte Meteor Systemarchitektur nach Watson \cite{meteor-architecture}}
  \label{fig:MeteorArchitektur}
\end{figure}

\subsection{Programmiersprache und Plattform}

Die Entwicklungsplattform wird Meteor sein, eine komplett in JavaScript
geschriebene Plattform für die Entwicklung von Webanwendungen. Meteor stellt
eine isomorphe API für Client- und Serverentwicklung bereit. Meteor folgt sieben
Designprinzipien \cite{meteor-7}, die interessierten Entwicklern eine Umgebung
für effizientes und schnelles Entwickeln von Webanwendungen ermöglichen soll.

\begin{quotation}

  \begin{description}

    \item[Datenübertragung]{Meteor schickt kein HTML über das Netzwerk, sondern
    nur Daten. Der Client rendert diese Daten dann}

    \item[Eine Sprache]{Sowohl Client-, als auch Serverkomponente werden in
    JavaScript geschrieben}

    \item[Datenbank überall]{Die Datenbank kann auf Client- und auf Serverseite
    mit den gleichen Methoden manipuliert werden.}

    \item[Verzögerungskompensierung]{Auf der Clientseite werden Daten
    vorgespeichert und Datenbankoperationen simuliert, um eine verzögerungsfreie
    Benutzererfahrung zu ermöglichen.}

    \item[Reaktivität über alle Schichten]{Mit Meteor ist Echtzeit der Standard.
    Alle Schichten, von der Datenbank bis zu den Templates, aktualisieren sich
    selbständig.}

    \item[Nutze das Ökosystem]{Meteor lässt sich einfach mit bestehenden
    Bibliotheken und Frameworks integrieren.}

    \item[Einfach bedeutet produktiv]{Die beste Art, etwas einfach aussehen zu
    lassen ist, etwas tatsächlich einfach zu machen. Die Hauptfunktionen von
    Meteor stellen einfache APIs bereit.}

  \end{description}

\end{quotation}

Die Wahl fiel auf Meteor, da ich mir insbesondere aufgrund der Einsprachigkeit,
der Isomorphie und der vollständigen Reaktivität Effizienzsteigerungen bei der
Entwicklung verspreche.

Zudem bietet Meteor die Möglichkeit, das Programm für verschiedene
Zielplattformen (derzeit Node.js, Android und iOS) zu kompilieren. So wird das
Bedienen der unterstützten Plattformen stark vereinfacht.

Meteor wurde auch deshalb ausgewählt, um Erfahrungen mit der Plattform zu
sammeln, so dass sich MTC ggf. auch strategisch auf diese Plattform stützen kann
und auch zukünftige Projekte mit dieser Technologie realisieren kann.

\section{Drei-Schicht-Architektur nach Meteor}

Aufgrund der besonderen Topologie der Meteor-Plattform ist es nicht ganz
einfach, die klassische Drei-Schicht-Architektur nach Balzert \cite{balzert} im
Entwurf zu realisieren. Die Schichten für Präsentation, Logik und Daten lassen
sich zwar auch hier wiederfinden, eine klassische Zuordnung von Klassen, deren
Abwandlungen sich dann in den verschiedenen Schichten wiederfinden, ist jedoch
nicht einfach abzubilden.

Meteor stellt für die verschiedenen Schichten bestimmte Werkzeuge zu Verfügung,
die im Folgenden kurz dargestellt werden sollen.

\begin{figure}[h]
  \includegraphics[width=0.9\textwidth]{gfx/meteor_3_schicht.pdf}
  \caption{Vereinfachte Meteor Drei-Schicht-Architektur}
  \label{fig:MeteorDreiSchichtArchitektur}
\end{figure}

\subsection{Präsentationssschicht}

Die Präsentationsaufgaben werden bei Meteor von der Templatebibliothek Blaze
\cite{blaze} übernommen. Darstellungselemente wie Webseiten, Formulare und
Navigationselemente werden als Templates im Spacebars-Format definiert.
Spacebars \cite{spacebars} akzeptiert eine HTML-artige Syntax, die mit einfachem
Markup für die Darstellungslogik angereichert ist.

Mit dem Spacebars-Markup lassen sich Bereiche an Funktionen binden, die
Daten für die Darstellung berechnen und bereitstellen.

Blaze registriert eigenständig Veränderungen der mit dem Template verbundenen
Daten, ermittelt die betroffenen Bereiche im Template und aktualisiert das
Document Object Model im Webbrowser, um die Veränderungen zu reflektieren.

Für die Darstellungslogik stellt Meteor für jedes Template ein Objekt
unter dem Namespace "`Template"' zur Verfügung. In diesem Objekt sind Objekte
und Funktionen enthalten, mit denen das Eventhandling und die Präsentationslogik
definiert werden kann. Die Logik für die Präsentation wird in Form von
Funktionen im Namespace Template.nameDesTemplates.helpers definiert.
Die Rückgabewerte dieser Funktionen können dann im betreffenden Template
mithilfe des Spacebars-Markup und des Templatenamens platziert werden.

\subsection{Logikschicht}

Für die Behandlung der Geschäftslogik bietet Meteor sogenannte
"`Meteor.methods"'. Diese Methoden sind global definiert und können durch die
Methode Meteor.call() aufgerufen werden. Meteor.methods sind der einzige Weg, um
Datenbankzugriffe durchzuführen.

Meteor.methods können sowohl auf dem Client als auch auf dem Server definiert
werden, ihr Verhalten ist unterschiedliche, je nach dem in welcher Umgebung sie
sich befinden.

Auf dem Server definierte Meteor.methods können vom Client aus aufgerufen
werden. Sie werden dann auf dem Server ausgeführt und liefern ggf. ein Ergebnis
zurück.

Werden Meteor.methods auf dem Client und dem Server definiert, verursacht der
Aufruf der  Methode auf dem Client parallel dazu auch einen Aufruf auf dem
Server. Lokal auf dem Client wird die Methode augenblicklich ausgeführt und eine
Simulation des Ergebnisses verwendet, um die Daten im Client verzögerungsfrei zu
aktualisieren. Ist der entfernte Aufruf auf dem Server beendet und hat ein
Ergebnis zum Client zurückgesandt, wird dieses das simulierte Ergebnis (vom
Anwender unbemerkt) ersetzen.

\subsection{Datenhaltungsschicht}

Die Datenhaltung ist in Meteor analog zur Logikschicht, sowohl auf dem Client,
als auch auf dem Server vorhanden.

Auf dem Server betreffen Datenbankzugriffe direkt die Datenbank, so wie man es aus
anderen Plattformen wie z.B. Java Spring gewohnt ist. Zur Verwendung kommt hier
der Node-MongoDB-Datenbanktreiber.

Auf dem Client existiert eine Replik der Datenbank, die durch eine
JavaScript-Implementierung des MongoDB-Datenbanktreibers namens "`Minimongo"'
\cite{minimongo} realisisert ist. Die an den Client replizierten Daten werden
dabei ggf. vom Server mittels Authentifizierungs- und Authorisierungsregeln
vorgefiltert, so dass nur Daten die für den verbunden Client bestimmt sind
gesendet werden.

Der Client hat keinen direkten Zugriff auf die Datenbank. Er kann nur Daten aus
seiner lokalen Replik lesen und alle Schreibzugriffe lösen Remote Procedure
Calls auf die API des Servers aus, so dass nur Servercode die Datenbank
manipulieren kann.

\section{Fachklassenmodell}

Aus den fachlichen Anforderungen in \autoref{ch:Pflichtenheft}
(Seite \pageref{ch:Pflichtenheft}) konnte ich ein Aktivitätsmodell ableiten, das ich
in Form eines Aktivitätsdiagramms (\autoref{fig:PNGActivity}, Seite
\pageref{fig:PNGActivity}) abgebildet habe.

\begin{figure}[H]
  \includegraphics[width=0.9\textwidth]{gfx/activity.pdf}
  \caption{Profil.NG - Aktivitätsmodell}
  \label{fig:PNGActivity}
\end{figure}

Die in diesem Diagramm wiederkehrenden Begriffe "`Kunde"', "`Monitoring"',
"`Query"' bildeten die Basis für das Fachklassenmodell
(\autoref{fig:PNGFachklassen}, Seite \pageref{fig:PNGFachklassen}).7

Hierbei ist anzumerken, dass in einer früheren Version die Klasse "`Occurance"'
noch nicht existierte. Als während der Analyse der Geschäftsprozesse jedoch die
Anforderung ermittelt wurde, dass Produktionsläufe einzelner
Überwachungsaufträge rückgängig gemacht oder für eine spätere Produktion
zurückgestellt werden sollten, wurde schnell klar, dass jedes
Produktionsereignis einzeln behandelt werden muss und sich das Konzept des
Berechneten Kalenders in
\autoref{ch:Lösungsansätze:Produktionsplan:BerechneterKalender} (Seite
\pageref{ch:Lösungsansätze:Produktionsplan:BerechneterKalender}) als nicht
ausreichend für die gestellten Anforderungen erwies.

Die Klasse "`DeliveredDoc"' leitete ich aus der Anforderung ab, dass keine
Patentdokumente doppelt an einen Kunden ausgeliefert werden dürfen. Deshalb
werden Objekte gespeichert, die ausgelieferte Dokumente repräsentieren und die
eindeutig einem bestimmten Monitoring zugeordnet werden.

\begin{figure}[H]
  \includegraphics[width=0.9\textwidth]{gfx/fachklassen.pdf}
  \caption{Profil.NG - Fachklassenmodell}
  \label{fig:PNGFachklassen}
\end{figure}

\section{Sequenzdiagramme}

Im folgenden werden Sequenzdiagramme angegeben, welche Funktionalitäten abbilden,
die nicht von Meteor-Standardfunktionen (wie z.B. Login/Logout) abgedeckt sind.

\subsection{[03] Kunde anlegen}
\label{ch:SystemEntwurf:03}

\begin{figure}[H]
  \includegraphics[width=0.9\textwidth]{gfx/03-kunde-anlegen.pdf}
  \caption{Anwendungsfall 03: Kunde anlegen}
  \label{fig:AF03}
\end{figure}

\subsection{[04] Kunde bearbeiten}
\label{ch:SystemEntwurf:04}

\begin{figure}[H]
  \includegraphics[width=0.9\textwidth]{gfx/04-kunde-bearbeiten.pdf}
  \caption{Anwendungsfall 04: Kunde bearbeiten}
  \label{fig:AF04}
\end{figure}

\subsection{[05] Überwachungsauftrag erstellen}
\label{ch:SystemEntwurf:05}

\begin{figure}[H]
  \includegraphics[width=0.9\textwidth]{gfx/05-uberwachungsauftrag-erstellen.pdf}
  \caption{Anwendungsfall 05: Überwachungsauftrag erstellen}
  \label{fig:AF05}
\end{figure}

\subsection{[06] Suchprofil erstellen}
\label{ch:SystemEntwurf:06}

\begin{figure}[H]
  \includegraphics[width=0.9\textwidth]{gfx/06-suchprofil-erstellen.pdf}
  \caption{Anwendungsfall 03: Suchprofil erstellen}
  \label{fig:AF06}
\end{figure}

\subsection{[07] Trefferlisten erstellen}
\label{ch:SystemEntwurf:07}

\begin{figure}[H]
  \includegraphics[width=0.9\textwidth]{gfx/07-trefferlisten-erstellen.pdf}
  \caption{Anwendungsfall 07: Trefferlisten erstellen}
  \label{fig:AF07}
\end{figure}

\subsection{[08] Trefferlisten anzeigen}

\begin{figure}[H]
  \includegraphics[width=0.9\textwidth]{gfx/08-trefferlisten-anzeigen.pdf}
  \caption{Anwendungsfall 08: Trefferlisten anzeigen}
  \label{fig:AF08}
\end{figure}

\subsection{[09] Ausgabeformate erstellen}

\begin{figure}[H]
  \includegraphics[width=0.9\textwidth]{gfx/09-ausgabeformate-erstellen.pdf}
  \caption{Anwendungsfall 09: Ausgabeformate erstellen}
  \label{fig:AF09}
\end{figure}

\subsection{[10.1] Produktion zurücksetzen (Trefferlisten)}

\begin{figure}[H]
  \includegraphics[width=0.9\textwidth]{gfx/10-1-produktion-reset-hitlist.pdf}
  \caption{Anwendungsfall 10.1: Produktion zurücksetzen (Trefferlisten)}
  \label{fig:AF10-1}
\end{figure}

\subsection{[10.2] Produktion zurücksetzen (Ausgabeformate)}

\begin{figure}[H]
  \includegraphics[width=0.9\textwidth]{gfx/10-2-produktion-reset-output.pdf}
  \caption{Anwendungsfall 10.2: Produktion zurücksetzen (Ausgabeformate)}
  \label{fig:AF10-2}
\end{figure}
