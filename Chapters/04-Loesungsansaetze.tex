% Chapter 1

\chapter{Lösungsansätze} % Chapter title

\label{ch:Lösungsansätze} % For referencing the chapter elsewhere, use \autoref{ch:introduction}

%----------------------------------------------------------------------------------------

Im bestehenden System DEPAROM-Profil ist die sogenannte "`Profilanwendung"' die
zentrale Steuerungsanwendung für die Operatoren. Es handelt sich bei dieser
Anwendung um eine monolithische Anwendung mit grafischer Benutzeroberfläche, die
viele verschiedene Aufgaben erfüllt. Hierzu gehören auch Aufgaben, die nicht
direkt mit der Produktion der Patentinformationslieferungen zu tun haben, wie
beispielsweise der Import der Rohdaten, die Indexierung und die
Datenbankverwaltung. Hieraus resultiert eine schwere Wartbarkeit.

Die neue "`Profilanwendung"' mit der Bezeichnung "`Profil.NG"' soll einen enger
gefassten Scope erhalten. Sie soll ausschliesßlich verwaltende und steuernde
Aufgaben erfüllen. Die eigentliche Arbeit wird durch einzelnen Webdienste
verrichtet.

Folgende Probleme müssen vorrangig gelöst werden.

\section{Produktionsplan}
\label{ch:Lösungsansätze:Produktionsplan}

Die verschiedenen Suchaufträge müssen in regelmäßigen Abstanden durchgeführt
werden. Hierzu wurde bisher durch ein Perlscript eine Textdatei generiert,
welche die einzelnen Produktionsdaten für jede Kalenderwoche enthält. Das
Perlscript kann nur Produktionsdaten etwa 12 Monate im voraus generieren. Läuft
der aktuelle Produktionsplan aus, muss das Skript neu angestoßen werden, mit den
Parametern für den neuen Produktionszyklus. Wird dies vergessen, kommt es zu
Fehlerzustaänden, die nur durch Entwickler behoben werden können.

Um diesen Umstand zu umgehen, soll ein wochenbasierter Kalender implementiert
werden, welcher die Produktionsereignisse verwaltet, so dass die Operatoren sich
nicht mehr um den Produktionsplan kümmern müssen.

Die Verwaltung der sich wiederholenden Ereignisse ist niht ganz trivial und ich
möchte einige Lösungsideen diskutieren.

\subsection{Der berechnete Kalender}
\label{ch:Lösungsansätze:Produktionsplan:BerechneterKalender}

Um die Datenhaltung möglichst simpel zu halten, zog ich zunächst ein Modell in
betracht, bei dem nur die Kalenderwoche des Beginns der Patentüberwachung sowie
das Zeitintervall für die Wiederholung gespeichert würde. Mit diesen Daten wäre
es möglich für jede Kalenderwoche die fälligen Suchaufträge zu berechnen.

Dieses Modell funktioniert leider nur gut, wenn keine Unregelmäßigkeiten im
Zeitablauf vorkommen. Möchte man einzelne Ereignisse verschieben oder aussetzen,
müsste man Ausnahmeobjekte speichern und diese bei jedem Anzeigen des
Produktionsplans abrufen, um ggf. die Regelereignisse zu ignorieren. Mir
erschien diese Vorgehensweise als zu kompliziert. Zu dem müssen Metadaten zu den
Ereignissen gespeichert werden, und es muss möglich sein, bereits geschehene
Ereignisse zurückzusetzen, damit man eine fehlerbehaftete
Patentinformationslieferung erneut produzieren kann.

Da die Anforderungen das Modell des berechneten Kalenders übersteigen, und
ohnehin zu jedem Produktionsereignis weitere Daten gespeichert werden müssen
verwarf ich dieses Modell zugunsten eines Modells, das ich "Persistierter
Kalender" nenne.

\subsection{Der persistierte Kalender}
\label{ch:Lösungsansätze:Produktionsplan:PersistierteKalender}

Bei diesem Modell wird der Beginn der Patentüberwachung gespeichert und auch
jedes Produktionsereignis als eigenes Objekt. Dies ermöglicht das
Speichern zusätzlicher Metainformationen, die für jedes Ereignis individuell
sein können. Teil der Anforderung ist bspw. auch, dass Produktionsläufe
zurückgesetzt werden können. Dies ist z.B. dann erforderlich, wenn unbemerkt
fehlerhafte Eingangsdaten verarbeteitet wurden, oder wenn Patente von einem
Patentamt depubliziert werden.

Der Vorteil dieser Lösung liegt in seiner Flexibilität. Es ist nun möglich,
Produktionsläufe zu verschieben, zu löschen oder zurückzustellen und es können
beliebige Metadaten zu jedem Produktionslauf gespeichert werden.

Bei der Erstellung eines Suchauftrages müssen allerdings alle
Produktionsereignisse über den Überwachungszeitraum berechnet, erstellt und
gespeichert werden. Dies kann abhängig von Dauer und Häufigkeit der
Patentüberachung zu einer großen Zahl an Datenbankoerationen führen. Zudem ist
noch ungeklärt, wie mit Suchaufträgen umgegangen werden soll, die kein
Ablaufdatum besitzen.

Vermutlich wird es aber auf einen Algorhitmus hinauslaufen, der die
Produktionsereignisse zunächst für einen festgelegten Zeitraum im Voraus
erstellt (bspw. für 12 Monate). Im Modell der Produktionsereignisse wird es dann
eine Routine geben die prüft, ob noch Produktionsereignisse bevorstehen und ggf.
weitere 12 Monate mit Produktionsereignissen füllen.

\section{Interserverkommunikation}
\label{ch:Lösungsansatze:Interserverkommunikation}

Die Kommunikation mit anderen Diensten über ein Netzwerk ist eine völlig neue
Problematik, die beim bisherigen System nicht auftrat. Da DEPAROM.NG aus
verschiedenen Diensten mit fest definierten Aufgaben besteht, muss Profil.NG
alle Aufgaben delegieren, die außerhalb ihres Scopes liegen. Hierzu gehören u.a.
die Patentsuche und die Erstellung der Ausgabeformate.

Diese Aufgaben laufen asynchron ab und können einige Zeit in Anspruch nehmen. Es
wird also erforderlich sein, eine Methode zur Verwaltung dieser Aufgaben zu
entwickeln. Ein gangbarer Weg wäre es, eine Work Queue zu implementieren, in
der die einzelnen Jobobjekte abgearbeitet werden können. Darüber hinaus sollten
die Jobobjekte in der Datenbank persistiert werden, damit sie auch einen
Serverneustart überleben und die Produktionsvorgänge konsistent bleiben.
