% Chapter 1

\chapter{Pflichtenheft} % Chapter title

\label{ch:Pflichtenheft} % For referencing the chapter elsewhere, use \autoref{ch:introduction}

%----------------------------------------------------------------------------------------

Im folgenden werden die Anforderungen an die Anwendung "`Profil.NG"' definiert.
Wie bei einem Pflichtenheft üblich sind die Anforderungen in "`Muss-"',
"`Kann-"' und "`Abgrenzungskritereien"' unterteilt. Im Anschluß folgt eine kurze
Beschreibung der Betriebsumgebung sowie eine Beschreibung der erforderlichen
Funktionalitäten anhand von User Stories.

\section{Zielsetzung}

Die Anwendung Profil.NG soll den Anwender (im Folgenden "`Operator"') dazu
befähigen, die Produktion Patentinformationslieferung im Rahmen des
Patentinformationsdienstes DEPAROM-Profil zu steuern und die Qualitätsstandards
des Produkts zu gewährleisten.

\subsection{Muss-Kriterien}

Profil.NG muss...

\begin{enumerate}
  \item{mit den Webbrowsern Safari, Chrome und Firefox benutzbar sein.}
  \item{eine grafische Benutzeroberfläche in Form einer Fat-Client Webanwendung
        haben.}
  \item{durch Benutzerkonten mit Login geschützt sein.}
  \item{Operatoren befähigen, Kundenkonten zu verwalten.}
  \item{Operatoren befähigen, Suchaufträge zu verwalten.}
  \item{Operatoren befähigen, die Patentinformationslieferungen zu produzieren.}
  \item{Operatoren befähigen, die Konsistenz der Patentinformationslieferungen
        zu gewährleisten, insbesondere aber keine Dokumente doppelt
        auszuliefern.}
\end{enumerate}

\subsection{Kann-Kriterien}

Profil.NG sollte...

\begin{enumerate}
  \item{den Operatoren einen Kalender zur Planung der Produktionsläufe
        bereitstellen.}
  \item{die Operatoren anhand von Statistiken und Histogrammen bei der
        Qualitätssicherung unterstützen.}
  \item{die Operatoren befähigen, Patentdokumente aus nichtregulären Quellen in
        die Patentinformationslieferungen einzufügen.}
  \item{den Operatoren eine transparente Übersetzungsfunktion bieten, die
        Suchanfragen von der DEPAROM-Query-Sprache in die
        Elasticsearch-Query-Sprache übersetzt, so dass die Operatoren nur mit
        DEPAROM-Queries arbeiten müssen.}
  \item{statistische Daten erheben, die für die Rechnungsstellung verwendet
        werden können.}
  \item{druckbare Seiten anzeigen können, die der Dokumentation der
        Produktionswoche dienen.}
\end{enumerate}

\subsection{Abgrenzungskriterien}

Profil.NG soll nicht...

\begin{enumerate}
  \item{eigene Dienste für die Verarbeitung von Patentiformationen
        bereitstellen.}
  \item{Ausgabeformate erstellen.}
  \item{Rechnungen erstellen.}
  \item{die Produktionsabläufe und Aktionen der Operatoren protokollieren, also
        eine elektronische Produktionsdokumentation erstellen.}
\end{enumerate}

\section{Produkteinsatz}

Profil.NG wird als Intranetanwendung im Firmennetzwerk betrieben. Nur ein
Anwender, der sich mittels Logindaten authentifiziert kann alle Funktionen der
Anwendung benutzen. Die Anwendung kann nur über das Intranet oder einen
VPN-Zugang zu selbigem erreicht werden.

\subsection{Zielgruppe}

Profil.NG wird von speziell geschulten Operatoren benutzt.

\subsection{Betriebsbedingungen}

Profil.NG muss zu den Bürozeiten im Intranet erreichbar sein. Die Anwendung wird
auf der firmeneigenen Infrastruktur in einer virtuellen Maschine betrieben
werden. Die  zu erwartende Last auf dieser Maschine wird aufgrund der geringen
Nutzerzahl (max. 2 gleichzeitige Benutzer) gering eingeschätzt. Auf eine
redundante Systemarchitektur mit mehreren Applikationsservern kann verzichtet
werden.

\subsection{Technische Produktumgebung}

Es wird empfohlen, die Webanwendung auf einem eigenen Server abzulegen. Davon
getrennt sollte die Datenbank (MongoDB) als Replikaset konfiguriert werden, und
auf mindestens zwei verschiedene Server verteilt werden. Hierdurch wird die
nötige Datensicherheit gewährleistet, auch wenn eine Serverinstanz ausfallen
sollte.

\subsubsection{Hardware}

Die Anwendung wird auf der firmeneigenen Virtualisierungsinfrastruktur
installiert. Diese bietet mehr als ausreichende Leistungsreserven, um die zu
erwartende Last auf Appliaktions- und Datenbankserver zu bewältigen.

\subsubsection{Software}

Für den Applikationsserver wird benötigt:

\begin{itemize}
  \item{Linuxbetriebssystem, z.B. Ubuntu}
  \item{Applikationsserver: Node.js}
  \item{Webserver, z.B. Nginx}
\end{itemize}

Für die Datenbankserver wird benötigt:

\begin{itemize}
  \item{Linuxbetriebssystem, z.B. Ubuntu}
  \item{Datenbanksystem: MongoDB}
\end{itemize}

\subsubsection{Produktschnittstellen}

Profil.NG bietet eine grafische Benuterschnittstelle, um Eingaben der Operatoren
entgegenzunehmen.

Das System verwendet Schnittstellen zu folgenden Diensten im DEPAROM.NG-Ökosystem:

\begin{itemize}
  \item{Simple Reference Store}
  \item{Collection Creator}
\end{itemize}

\section{Produktübersicht}

Als Produktübersicht soll der Arbeitsablauf (\autoref{fig:DEPOPSWorkflow},
Seite \pageref{fig:DEPOPSWorkflow}) der Operatoren dienen. Diesen wird Profil.NG
ermöglichen und unterstützen.

\begin{figure}[h]
  \includegraphics[height=0.9\textheight]{gfx/operator_workflow_vertikal.pdf}
  \caption{DEPAROM Operatoren Arbeitsablauf}
  \label{fig:DEPOPSWorkflow}
\end{figure}

\section{Produktfunktionen}

Anhand eines Use-Case-Diagramms (\autoref{fig:PNGUseCases}, Seite
\pageref{fig:PNGUseCases}) soll der Funktionsumfang skizziert werden. Im
Anschluss folgt die Beschreibung der Anwendungsfälle in Form von User Stories.

\begin{figure}[h]
  \includegraphics[width=0.9\textwidth]{gfx/use_cases.pdf}
  \caption{Profil.NG-Anwendungsfälle}
  \label{fig:PNGUseCases}
\end{figure}

\subsection{User Stories}

Die hier beschriebenen User Stories folgen einem bestimmten Format, das den
Autor, aber auch den Leser

\begin{enumerate}
  \item{in die Gedankenwelt des jeweiligen Anwenders versetzen soll}
  \item{die gewünschte Aktivität des Anwenders beschreiben soll}
  \item{die zugrundeliegende Motivation des Anwenders herausstellen soll.}
\end{enumerate}

Dies soll beim Systementwurf und der Implementierung dabei helfen, jeweil im
Scope der Anwendungsfälle zu bleiben und bei der Diskussion über unscharfe
Anforderungen schon einen Hinweis auf die gewünschte Richtung geben soll.

Das Template entspricht folgendem Format von Mike Cohn
\cite{user-stories-format}:

\begin{quotation}
  Als <Anwendertyp> möchte ich <ein Ziel erreichen>, damit <ich einen Nutzen habe>
\end{quotation}


\subsubsection{[01] Im System authentifizieren}

Als Operator

möchte ich mich bei Profil.NG anmelden,

damit ich meine Produktionsaufgaben erfüllen kann.

\subsubsection{[02] Vom System abmelden}

Als Operator

möchte ich mich von Profil.NG abmelden,

damit ich keinen unberechtigten Zugriff von meiner Arbeitsstation erlaube.

\subsubsection{[03] Kunde anlegen}

Als Operator

möchte ich ein neues Kundenprofil anlegen,

damit ich Überwachungsaufträge für den Kunden erstellen kann.

\subsubsection{[04] Kunde bearbeiten}

Als Operator

möchte ich ein Kundenprofil bearbeiten,

damit ich veränderte Stammdaten meines Kunden übertragen kann.

\subsubsection{[05] Überwachungsauftrag anlegen}

Als Operator

möchte ich einen Überwachungsauftrag in den Stammdaten meines Kunden anlegen,

damit ich das Suchprofil für meinen Kunden erstellen kann.

\subsubsection{[06] Suchprofil erstellen}

Als Operator

möchte ich ein Suchprofil im Überwachungsauftrag meines Kunden erstellen,

damit ich die Patentinformationslieferung für meinen Kunden zusammenstellen kann.

\subsubsection{[07] Trefferlisten erstellen}

Als Operator

möchte ich auf einer Übersicht der vorhandenen Überwachungsaufträge die zu produzierenden
Aufträge auswählen für sie Trefferlisten ermitteln lassen,

damit ich die Qualitätskontrolle durchführen kann und die Ausgabeformate erstellen kann.

\subsubsection{[08] Trefferlisten anzeigen}

Als Operator

möchte ich die Anzahl der Treffer und die Trefferlisten der zu produzierenden
Überwachungsaufträge einsehen,

damit ich die Qualitätskontrolle durchführen kann.

\subsubsection{[09] Ausgabeformate erstellen}

Als Operator

möchte ich Trefferlisten von Überwachungsaufträgen auswählen und für diese die
Ausgabeformate erstellen,

damit ich sie anschließend für die Konfektionierung weiterverarbeiten kann.
