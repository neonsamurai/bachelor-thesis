% Chapter 1

\chapter{Aufgabenstellung} % Chapter title

\label{ch:aufgabenstellung} % For referencing the chapter elsewhere, use \autoref{ch:introduction}

%----------------------------------------------------------------------------------------

Die sogenannte Profil-Anwendung dient den Operatoren als zentrale
Steuersoftware, um die Verarbeitung der Patentdaten zu vorausgewählten
Patentsammlungen sicherzustellen. Die monolithische Anwendung wurde seit Beginn
ihres Bestehens in der Mitte der 1990er Jahre ständig erweitert und modifiziert,
um die reibungslose Verarbeitung der durch die Patentämter fortlaufend
veränderten Patentdatenformate sicherstellen zu können.

Die Aufgabe dieser Arbeit besteht darin, diese Anwendung mit 20 Jahre alter
Codebase in das 21. Jahrhundert zu befördern. Hierfür sollen die vorhandenen
Geschäftsprozesse analysiert, die für den zukünftigen Betrieb notwendigen Teile
isoliert und in eine moderne und schlanke Spezifikation überführt werden.

Diese Spezifikation in Form von User Stories und Datenmodellen soll anschließend
in einem agilen Entwicklungsprozess umgesetzt werden. Regelmäßige Rücksprachen
mit den zukünftigen Anwendern der Software sollen sicherstellen, dass die
Implementierung die korrekte Interpretation der Spezifikation darstellt und
Abweichungen kurzfristig korrigiert werden können.

Als Ergebnis soll eine Webanwendung entstehen, welche die Operatoren befähigt,
die regelmäßigen Datenlieferungen für die Kunden zu erstellen.

\section{Abgrenzung}

Dabei soll Profil.NG lediglich verwaltende und steuernde Aufgaben erfüllen und
die  umliegenden Dienste des neuen Profilökosystems verwenden, um die
Datenlieferungen zu erstellen.

Es ist ausdrücklich nicht Teil dieser Arbeit, weitere Dienste zu erstellen oder
zu modifizieren. Die zu erstellende Anwendung soll ausschließlich die
Regelproduktion sicherstellen. Für einige Kunden bestehen Speziallösungen, in
deren Rahmen Patentinformationen in bestimmten Formaten ausgeliefert werden.
Diese Sonderwege sind nicht Teil dieser Aufgabe.
