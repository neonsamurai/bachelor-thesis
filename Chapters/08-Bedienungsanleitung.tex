% Chapter 1

\chapter{Bedienungsanleitung} % Chapter title

\label{ch:bedienungsanleitung} % For referencing the chapter elsewhere, use \autoref{ch:introduction}

%----------------------------------------------------------------------------------------

\section{Installationsvorraussetzungen}

Für die Installation der Entwicklerversion wird benötigt:

\begin{description}
  \item[Betriebssystem]{Mac OSX oder Linux}
  \item[curl]{Um das Installationsscript zu beziehen und auszuführen}
  \item[Internetverbindung]{Die Installation ist nur über das Internet möglich}
\end{description}

Die Installation einer Produktivumgebung ist komplexer und soll nicht Gegenstand
dieser Arbeit sein.

\section{Installation und Start}

Installieren Sie die Meteorplattform mittels folgendem Shell-Befehl:

\begin{verbatim}
  $> curl https://install.meteor.com/ | sh
\end{verbatim}

Kopieren sie das Verzeichnis \textit{Quellcode/profil-ng} von dem beiliegenden
Datenräger auf ein beliebiges Festplattenlaufwerk ihres Computers und wechseln
in das betreffende Verzeichnis.

Starten Sie die Anwendung mit dem Befehl

\begin{verbatim}
  $> meteor
\end{verbatim}

Meteor installiert jetzt alle Abhängigkeiten und startet anschließend den
Entwicklungsserver. Die Webanwendung ist unter der URL
\textit{http://localhost:3000} über einen Webbrowser erreichbar.

\section{Unterstützte Webbrowser}

Folgende Webbrowser wurden getestet und offiziell unterstützt:

\begin{itemize}
  \item{Google Chrome}
  \item{Apple Safari}
\end{itemize}

Firefox wurde zwar nicht getestet, es ist aber wahrscheinlich, dass
Profil.NG problemlos mit Firefox zusammenarbeitet.

Microsoft Internet Explorer wurde in keiner Version getestet. Es werden
keinerlei Garantien für die Verwendung übernommen.


\section{Benutzerhandbuch}

Profil.NG erlaubt die Verwaltung der Produktionsabläufe für DEPAROM-Profil,
sowie der für die Produktion relevanten Stammdaten der Kunden. Hierzu gehören
Kundendaten, Überwachungsaufträge und Queries.

\subsection{Vor dem Erstbetrieb}

Da alle Funktionen hinter einem Login verborgen sind, muss zunächst ein
Benutzerkonto angelegt werden. Den Link zum Registrierungsformular finden Sie
unterhalb des Anmeldeformulars.

\begin{figure}[H]
  \includegraphics[width=0.9\textwidth]{gfx/login-register.pdf}
  \caption{Link zum Registrierungsformular}
  \label{fig:login-register}
\end{figure}

Füllen Sie das Registrierungsformular aus und klicken auf die Schaltfläche
\textit{Register Now!}. Ein Benutzerkonto wird für Sie angelegt und Sie werden
augenblicklich mit selbigem angemeldet.

\begin{figure}[H]
  \includegraphics[width=0.9\textwidth]{gfx/register-form.pdf}
  \caption{Registrierungsformular}
  \label{fig:register-form}
\end{figure}

\subsection{Kunde anlegen}

Klicken Sie in der Navigationsleiste auf die Schaltfläche \textit{Manage
customers}. Es erscheint eine Übersicht der im System befindlichen Kundenkonten.

\begin{figure}[H]
  \includegraphics[width=0.9\textwidth]{gfx/add-customer.pdf}
  \caption{Schaltfläche 'Add customer'}
  \label{fig:add-customer}
\end{figure}

Klicken Sie auf die Schaltfläche \textit{Add customer}, um das Formular zum
Erstellen eines neuen Kunden zu öffnen. Füllen Sie das Formular aus. Mit einem
Stern markierte Felder sind Pflichtfelder. Bitte lassen Sie diese nich leer.
Sind Sie mit Ihrer Eingabe zufrieden, klicken Sie auf die Schaltfläche
\textit{Save customer}

\begin{figure}[H]
  \includegraphics[width=0.9\textwidth]{gfx/add-customer-form.pdf}
  \caption{Formular für das Erstellen eines neuen Kunden}
  \label{fig:add-customer-form}
\end{figure}

\subsection{Überwachungsauftrag anlegen}

Klicken Sie in der Navigationsleiste auf die Schaltfläche \textit{Manage
customers}. Es erscheint eine Übersicht der im System befindlichen Kundenkonten.

Klicken Sie auf den Namen eines Kunden, um die Ansicht zum Bearbeiten des Kunden
zu öffnen. Siehe \autoref{fig:add-customer}, Seite \pageref{fig:add-customer}.

Scrollen Sie in der Ansicht zum Bearbeiten des Kunden nach unten, bis die
Schaltfläche \textit{Add monitoring} erscheint. Klicken Sie auf diese
Schaltfläche.

\begin{figure}[H]
  \includegraphics[width=0.9\textwidth]{gfx/add-monitoring.pdf}
  \caption{Schaltfläche 'Add monitoring'}
  \label{fig:add-monitoring}
\end{figure}

Füllen Sie das Formular zum Anlegen eines neuen Überwachungsauftrags aus.
Klicken Sie auf die Schaltfläche \textit{Save monitoring}, wenn Sie mit ihren
Eingaben zufrieden sind.

\begin{figure}[H]
  \includegraphics[width=0.9\textwidth]{gfx/add-monitoring-form.pdf}
  \caption{Formular für das Erstellen eines neuen Überwachungsauftrags}
  \label{fig:add-monitoring-form}
\end{figure}

\subsection{Suchanfrage anlegen}

Klicken Sie in der Navigationsleiste auf die Schaltfläche \textit{Manage
customers}. Es erscheint eine Übersicht der im System befindlichen Kundenkonten.

Klicken Sie auf den Namen eines Kunden, um die Ansicht zum Bearbeiten des Kunden
zu öffnen. Siehe \autoref{fig:add-customer}, Seite \pageref{fig:add-customer}.

Klicken Sie auf die \textit{Profile \#} eines Monitorings, um die Ansicht zum
Bearbeiten des Monitorings zu öffnen. Siehe \autoref{fig:add-monitoring}, Seite
\pageref{fig:add-monitoring}.

Scrollen Sie in der Ansicht zum Bearbeiten des Monitorings nach unten, bis die
Schaltfläche \textit{Add query} erscheint. Klicken Sie auf diese Schaltfläche.

\begin{figure}[H]
  \includegraphics[width=0.9\textwidth]{gfx/add-query.pdf}
  \caption{Schaltfläche 'Add query'}
  \label{fig:add-query}
\end{figure}

Füllen Sie das Formular zum Anlegen eines neuen Queries aus.
Klicken Sie auf die Schaltfläche \textit{Save query}, wenn Sie mit ihren
Eingaben zufrieden sind.

\begin{figure}[H]
  \includegraphics[width=0.9\textwidth]{gfx/add-query-form.pdf}
  \caption{Formular für das Erstellen eines neuen Query}
  \label{fig:add-query-form}
\end{figure}
