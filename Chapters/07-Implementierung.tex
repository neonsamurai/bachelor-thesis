% Chapter 1

\chapter{Implementierung} % Chapter title

\label{ch:implementierung}

%-------------------------------------------------------------------------------

Die Implementierung erfolgte auf Basis der Plattform Meteor. Da Meteor die aus
anderen Plattformen wie Java Enterprise, ASP.net oder Python/Django bekannte
Trennung von Client und Server auflöst, war es nötig in der
Implementierunsarbeit neue Wege zu finden, um eine saubere Softwarearchitektur
zu ermöglichen. Die  wesentlichen Maßnahmen hierzu, sowie die generelle
Vorgehensweise sollen in diesem Kapitel diskutiert werden.

Der Stand der Implementierung zum Zeitpunkt der Erstellung dieser Arbeit ist
unvollständig. Die Analyse der Anforderungen sowie der Entwurf des Systems wurde
immer wieder verzögert. Behinderungsfaktoren waren zum Einen
Terminschwierigkeiten mit den beteiligten Produktverantwortlichen, zum anderen
zunächst unscharfe und später wechselnde Anforderungen an das System. Dies
führte schlussendlich dazu, dass für die Implementierung nur etwa vier Wochen zu
Verfügung standen.

\section{Meteorspezifika}

Die Meteorplattform kommt mit einem eigenen Buildsystem, welches den Quellcode
packt, minimiert und für die Integration auf einem Node.js-Server bereit macht.
Damit dieses System wie erwartet funktioniert muss der Entwickler einigen
Konventionen bei der Wahl der Verzeichnisse beachten, da bestimmte Verzeichnisse
nur auf dem Client, nur auf dem Server oder aber auf beiden Seiten ausgeliefert
werden. Zudem werden die Verzeichnisse in einer bestimmten Reihenfolge geladen,
was bei Codefragmenten wichtig werden kann, die von einander Abhängig sind.

\begin{description}
  \item[server]{Wird nur an den Server ausgeliefert}
  \item[client]{Wird nur an de Client ausgeliefert}
  \item[tests]{Wird nur in der Testkonfiguration ausgeliefert}
  \item[lib]{Wird in einer Verzeichnisebene zuerst geladen}
\end{description}

Alle übrigen Verzeichnisse werden auf Client und Server ausgeliefert. Dateien
und Verzeichnisse werden, beginnend mit der tiefsten Verzeichnisebene, in
alphabetischer Reihenfolge geladen.

Ich folgte bei meiner Implementierung den Konventionen aus dem "`Unofficial
Meteor FAQ"' \cite{meteor-faq}, welche hier kurz als Beispielverzeichnisbaum
dargestellt werden soll.

\section{Vorgehensweise}

Um die Reihenfolge der zu implementierenden Anforderungen zu bestimmen, nahm ich
die erstellten User Stories und sortierte sie zunächst nach den wichtigsten bzw.
am schwierigsten zu implementierenden User Stories. Die mir am wichtigsten
erscheinende User Story ist \nameref{ch:SystemEntwurf:07} (Seite
\pageref{ch:SystemEntwurf:07}), da die Trefferliste den eigentlichen Mehrwert
der Dienstleistung Deparom-Profil und somit den zentralen Use Case darstellt.

Damit diese User Story realisiert werden kann identifizierte ich andere User
Stories, die als Vorraussetzung gelten.

\begin{enumerate}
  \item{\nameref{ch:SystemEntwurf:03}, Seite \pageref{ch:SystemEntwurf:03}}
  \item{\nameref{ch:SystemEntwurf:04}, Seite \pageref{ch:SystemEntwurf:04}}
  \item{\nameref{ch:SystemEntwurf:05}, Seite \pageref{ch:SystemEntwurf:05}}
  \item{\nameref{ch:SystemEntwurf:06}, Seite \pageref{ch:SystemEntwurf:06}}
\end{enumerate}

Bei der eigentlichen Implementierungsarbeit folgte ich den Prinzipien der
testgetriebenen Entwicklung \cite{tdd}, bei der zunächst ein programmatischer
Test die gewünschte Funktionalität prüft. Natürlicherweise scheitert dieser
Test, da die notwendige Implementierung fehlt.

Anschließend wird die fehlende Implementierung erarbeitet, bis der Test
erfolgreich verläuft.

Man beginnt nun mit dem nächsten Test und macht immer so weiter.

\subsection{Testgetriebene Entwicklung mit Meteor}

Für die testgetriebene Entwicklung stellt Meteor ein Testframework namens
Velocity \cite{velocity} bereit. Das Framework nutzt die Reaktivität von Meteor,
überwacht den Quellcode und lässt alle vorhandenen Tests ablaufen, sobald eine
geänderte Quelldatei gespeichert wird. Hierdurch kann sich der Entwickler auf
die Implementierung konzentrieren, da das Neuladen und Testen der Anwendung vom
Framework automatisch durchgeführt wird. Als Entwickler muss man nur noch die
Testergebnisse überprüfen.

Velocity unterstützt verschiedene Testbibliotheken. Ich verwendete Jasmine 2.1
\cite{jasmine}, eine Bibliothek die Unit und Integration Tests ermöglicht.
Jasmine ermöglicht das Schreiben von Tests sowohl für den Client, als auch für
den Server.

\section{Tests}

Dem Testen der Anwendung wurde viel Aufmerksamkeit und Entwicklungszeit
gewidmet. Aus diesem Grunde sollen einige Tests im Detail besprochen werden, um
die Besonderheiten beim Testen von Meteor-Applikationen aufzuzeigen.

\subsection{Unit Tests}

Bei den Tests konzentrierte ich mich vor allem auf die das Fachklassenmodell.
Eine vollständige Abdeckung meines Quellcodes mit Tests ist zwar erstrebenswert,
im Hinblick auf die Zeitbeschränkungen jedoch nicht zielführend.

Exemplarisch soll hier die Unit-Testsuite für die Customers-Klasse diskutiert
werden.

\lstinputlisting[caption=Unit Tests für Customers-Klasse]{/Users/timjagodzinski/src/don.ng/profil-ng/tests/jasmine/server/unit/customerModelSpec.js}

Ab Zeile 7 im obigen Listing wird mit der Funktion \textit{spyOn} ein sog. Spy
definiert. Ein Spy überwacht ein Objekt und protokolliert die Zugriffe auf das
Objekt. Auf das Protokoll kann zugegriffen werden, um das Verhalten der
Anwendung zu testen. Darüber hinaus ist es möglich, Funktionen zu definieren,
die aufgerufen werden, wenn ein überwachtes Objekt auf eine bestimmte Art und
Weise aufgerufen wird.

Diese Funktionalität wird im obigen Listing verwendet, um den Callback des
Aufrufs von Meteor.call() zu simulieren. Dies wird durch die Verkettung des
\textit{spyOn}-Aufrufs mit den Methoden \textit{.and.callFake()} erreicht, wobei
\textit{callFake} als Parameter die gewünschte Funktion entgegennimmt.

In der vorliegenden Implementierung wird nur der \textit{callback} mit den
Argumenten \textit{null}, \textit{1} aufgerufen. So wird simuliert, dass die
Meteor.method \textit{customerSave} ein leeres \textit{error}-Objekt und die ID
des neu hinzugefügten Customer-Objekts zurückgibt.

Die folgenden Zeilen bis Zeile 20 definieren ein Testdatenobjekt, anhand dem die
Konstruktorfunktion in Zeile 24 aufgerufen wird.

Ab Zeile 26 wird dann überprüft, ob der Inhalt der Felder des
\textit{customer}-Objekts denen des Testdatenobjekts entspricht. Dabei
referenziere ich auf das Testdatenobjekt und vermeide Verwendung von Literalen,
um die Vergleiche zu machen. Dies soll Fehler durch Tippfehler vermeiden.

Eine frühere Version dieses Tests nutzte zum Prüfen der Feldwerte des
\textit{customer}-Objekts Literale. Dabei unterlief mir ein Tippfehler und der
Test schlug fehl. Daraufhin führte ich eine Refakturierung auf die oben
abgebildete Version durch und vermied es auch in den weiteren Tests, Literale
zur Prüfung zu verwenden.

\subsection{Integration Tests}

Um das Zusammenspiel der einzelnen Komponenten der Anwendung zu testen, bedient
man sich der Integrationstests. Diese werden verwendet, um die Schnittstellen
der verwendeten Module zu testen, aber auch um das Zusammenspiel von Front- und
Backend zu testen, sowie das Verhalten der Clientapplikation in Bezug auf
Anwenderinteraktion.

Bei der Implementierungsarbeit für Profil.NG nutzte ich Integrationstests, um die
Routenkonfiguration zu testen. Hierbei war mir besonders wichtig sicherzustellen,
dass geschützte Routen nur für angemeldete Benutzer sichtbar sind.

Darüber hinaus habe ich die Beschaffenheit von Templates, insbesondere von
Formularen überprüft, sowie das Eventhandling auf den Routen.

Exemplarisch sollen hier die Tests für die Routen und Controller diskutiert
werden, welche für das Erstellen und Bearbeiten von Customer-Objekten zuständig
sind.

Da Integrationstests auf einer echten Instanz der Anwendung laufen, ist es
notwendig Testdaten vorzubereiten, die von der Anwendung verwendet werden
können. Insbesondere ist es nötig, ein Benutzerkonto anzugeben, damit sich der
virtuelle Anwender bei der Applikation authentifizieren kann.

Zu diesem Zweck wird eine Funktion definiert, welche aufgerufen wird, wenn die
Anwendung gestartet in der Testkonfiguration gestartet wird. Im folgenden
Listing ist zu sehen, wie des Testbenutzerkonto in der Funktion
\textit{loadDefaultFixtures()} definiert wird.

\lstinputlisting[firstline=50,caption=Fixtures für Integrationstests]{/Users/timjagodzinski/src/don.ng/profil-ng/tests/jasmine/server/integration/fixtures/database-fixture.js}

Die Testsuite für die Route, die für das Hinzufügen von Customer-Objekten
verwendet wird soll im Folgenden beschrieben werden.

\lstinputlisting[caption=Integrationstest für Kunde-Speichern-Formular]{/Users/timjagodzinski/src/don.ng/profil-ng/tests/jasmine/client/integration/customers/customersAddSpec.js}

Als erstes wird geprüft, ob die geforderte Route unter dem geforderten Namen
\textit{customersAdd} existiert und ob diese zur URL \textit{/customers/add}
führt.

\begin{lstlisting}[caption=Test auf Existenz der Route 'customersAdd']
  describe("route", function() {
    it("should be at path '/customers/add'", function() {
      expect(Router.routes["customersAdd"].path()).toBe(
        "/customers/add");
    });
  });
\end{lstlisting}

Dieser Test ist wichtig, um die generelle Routenkonfiguration der Anwendung zu
validieren. Falsche Routen, Pfade oder geänderte Routen sind häufige
Fehlerquellen in meiner Entwicklungsarbeit gewesen. Insbesondere bei steigender
Komplexität der Anwendung ist eine einwandfreie Routenkonfiguration wichtig, um
Doppelbelegungen von URLs oder die falsche Verwendung von Routen zu vermeiden.

Der eigentliche Test wird in der Testsuite \textit{view} beschrieben. Anzumerken ist hier
zunächst der Block mit den Aufrufen von \textit{beforeEach()} und \textit{afterEach} der im
folgenden Listing dargestellt ist.

\begin{lstlisting}[caption=Vor- und Nachtestkonfiguration]
  beforeEach(function(done) {
      Meteor.loginWithPassword("ops@profil.ng", "start1234",
        function(
          error) {
          done();
        });
    });

    beforeEach(function(done) {
      Router.go("/customers/add");
      Tracker.afterFlush(done);
    });

    beforeEach(waitForRouter);

    afterEach(function(done) {
      Meteor.logout(function() {
        done();
      });
    });
\end{lstlisting}

Die \textit{beforEach()}-Aufrufe werden vor jedem Test ausgeführt. Sie enthalten
für gewöhnlich Konfiguration und Abläufe, die für die kommenden Test wichtig
sind. In der vorliegenden Implementierung meldet sich der Testclient mit den
Nutzerdaten an, die in der Fixture schon definiert wurden. Anschließend wird der
Router aufgerufen und ändert die URL auf den vorher getesten Wert. Mit
\textit{beforeEach(waitForRouter);} wartet der Testclient darauf, dass die Seite
vollständig geladen ist, um  sicherzustellen dass der DOM vollstaändig ist. So
wird gewährleitet, dass die folgenden Tests, welche nach bestimmten
DOM-Elementen suchen auch erfolgreich ablaufen können.

Der Rest der Testsuite befasst sich mit dem eigenlichen Formular und überprüft,
ob alle erforderlichen Formularfelder und Schaltflächen vorhanden sind.

Mit Integrationstests kann auch das Verhalten der Anwendung bei Benutzerinteraktion
getestet werden. Hierfür soll zur Demonstration ein Auszug aus der Testsuite des
Controllers für die Route \textit{customersAdd} dienen.

\begin{lstlisting}[caption=Test des Verhaltens bei Benutzereingabe]
  it("should save the form contents to a new customer object on submit",
		function() {
			var customer;

			fillForm();
			spyOn(Customers, "insert");

			$(".ui.form.segment").form().submit();
			expect(Customers.insert).toHaveBeenCalled();
			expect(Customers.insert).not.toThrow();
		});
\end{lstlisting}

Mit der Hilfsfunktion \textit{fillForm()} wird zunächst das HTML-Formular mit
Daten gefüllt. Dies geschieht unter Zuhilfname der DOM-Manipulationsbibliothek
jQuery. Anschließend wird ein Submit-Event mittels jQuery ausgelöst, der
wiederum die Datenbankoperation zum Speichern des Customer-Objekts auslösen
soll. Das verhalten wird mit den beiden \textit{expect()}-Aufrufen überprüft.

Bei dieser Art Test ist es wichtig, das gewünschte Ergebnis abzutesten. In
diesem Fall der Aufruf der Insert-Methode auf der Customers-Collection. Der Weg
dort hin, also die eigentliche Implementierung sollte nicht abgetestet werden.
Dies würde dazu führen, dass bei Änderungen in der Implementierung der Test ggf.
scheitert, obwohl die Implementierung immer noch das gewünschte Ergebnis
liefert. Man spricht in diesem Fall von sogenannten "`brittle"' Tests
\cite{brittle-tests}.

\section{Fachklassen}

Sowohl auf Client als auch auf dem Server existiert das eigentliche Model in
Form einer Konstruktorfunktion, welche das jeweilige Model beschreibt. Objekte
die aus diesen Konstruktorfunktion erstellt werden rufen ihrerseits
Meteor.methods auf, wenn sie Datenbankoperationen veranlassen sollen. Diese
werden zum Zwecke der leichteren Zuordnung direkt neben den
Konstruktorfunktionen, also in der gleichen Datei definiert.

Die verschiedenen Modelle der Geschäftslogik sind im Verzeichnis \textit{models}
gesammelt.

Die Modelle enthalten vor allem CRUD-Operationen und sollen wegen ihrer
Einfachheit hier nicht näher erlautert werden.

\section{Publications und Subscriptions}

Meteor verwendet einen besonderen Mechanismus, um Daten vom Server zu Beziehen.
Da Meteor nicht dem klassischen Zyklus von Anfrage und Antwort des
HTTP-Protokolls folgt, sondern Daten per Websockets direkt an den Client
überträgt ist es erforderlich, Endpunkte auf dem Server zu definieren, an denen
sich ein Client anmelden kann, um Daten zu beziehen. Diese Endpunkte werden bei
Meteor \textit{Publications} genannt. Mittels einer Publication wird eine
Funktion  definiert welche einen oder mehrere Datenbankcursors zurückliefert,
welche dann vom Client genutzt werden können, um Daten abzurufen.

Die Publications für das Customer-Modell sehen wie folgt aus:

\lstinputlisting[caption=Definition der Publication für die Customers-Collection]{server/publications/customersPublications.js}

Ein Client kann sich an diese Publications anmelden. Dafür muss der Methode
\textit{Meteor.subscribe()} der Name der Publication, sowie ggf. nötige
Parameter übergeben werden. Als Rückgabewert erhält man einen
Subscription-Handle, mittels dem man dann die eigentlichen Daten beziehen kann.

Dieses Modell verleitet zunächst dazu, sehr weit gefasste Publications zu
schreiben, damit alle Daten, die zu einem Benutzer gehören auf den Client
synchronisiert werden, und dieser mit allen Daten lokal Arbeiten kann. Bei kleineren
Datenmengen ist dies noch ein gangbarer Weg. Aber bei großen Datenmengen wird diese
Vorgehensweise zu einem Performanceproblem.

Auf der Clientseite kann der initiale Aufruf der Anwendung mit einer sehr langen
Ladezeit verbunden sein, da erst alle Daten synchronisiert sein müssen, bevor
die Anwendung gestartet werden kann. Auf der Serverseite führen weit gefasste
Publications zu einem sehr großen Arbeitsspeicherverbrauch, da der
Node.js-Server alle Daten im Speicher behält, die in einer Pulication an einen
bestimmten Client ausgeliefert wurden. Dies ist notwendig, damit der Server bei
Änderungen in der Datenbank einen Abgleich ausführen kann, damit nur ein
Datenpatch und nicht der komplette Datensatz an den Client geschickt werden
müssen.

Aus diesem Grunde gilt der Grundsatz der Datensparsamkeit beim Entwurf von
Publications und beim Einsatz von Subscriptions. In der Implementierung der
Routecontroller wurde deshalb darauf geachtet, nur Subscriptions anzumelden, die
für die aktuelle Route relevant sind. Siehe dazu folgendes Listing.

\begin{lstlisting}[caption=Auszug aus Routecontroller für Route 'customers']
Router.route('/customers', function() {
  	this.render('Customers');
  }, {
  	name: 'customers',
  	waitOn: function() {
  		Meteor.subscribe("getCustomersList");
  	},
  	data: function() {
  		return {
  			customers: Customers.find().fetch()
  		};
  	}
});
\end{lstlisting}
